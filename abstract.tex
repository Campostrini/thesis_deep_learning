\chapter*{Abstract}
\addcontentsline{toc}{chapter}{Abstract}
\thispagestyle{plain}

Forecasting precipitation at very short lead times is a task that is difficult to address through numerical weather prediction models due to the great amount of computing power required to solve the physical equations that govern the behaviour of the atmosphere in a timely fashion. The advances in the field of deep learning and the large amount of historical data of radar precipitation maps that are openly available make the precipitation nowcasting problem a good candidate for deep learning solutions. To assess the potential of the U-Net neural network architecture at the 1hr time scale a comparison is made of its performance to persistence, i.e. the assumption that the prediction is equal to the last available frame. The problem is structured as a multi-class classification task over 4 precipitation intensity classes. The model takes 6 input images of 1h cumulative precipitation of the RW product from the RADar Online ANeichung (RADOLAN) dataset of the German weather service (DWD) and an area of 256km$\times$256km over northern Germany is selected. The task is to predict the precipitation classes of the pixels of the RADOLAN image at 6h from the first input image. It is found that the model performs better than persistence at higher precipitation rates. For rainfall rates in [0mm/h, 0.1mm/h) the model has a Critical Success Index (CSI) of 0.853 compared to 0.893 of persistence. For rainfall rates in [0.1mm/h, 1mm/h), [0.1mm/h, 1mm/h) and [0.1mm/h, $\infty$mm/h) the model has better CSI than persistence (0.292 vs 0.276, 0.253 vs 0.182 and 0.191 vs 0.128 respectively). For the same classes the Heidke Skill Score (HSS) absolute performance diminishes with increasing precipitation rate but gets better relative to persistence (0.579 vs 0.583, 0.438 vs 0.441, 0.404 vs 0.314 and 0.321 vs 0.229). Finally, both a higher average HSS for the model compared to persistence (0.436 compared to 0.392) and a higher average CSI (0.397 compared to 0.370) are found. A visual inspection of the predictions suggests that the neural network is able to infer the general structure of the precipitation field but has a tendency to predict higher rain rates. It can be concluded that the results are encouraging and that the application of the U-Net architecture to 1h cumulative precipitation maps should be further investigated.
