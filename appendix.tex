\chapter{Software and Hardware}\label{appA}
\thispagestyle{plain}

The software was written in \textit{Python 3.8}. The RADOLAN dataset was handled through \textit{wradlib} \citep{Muhlbauer2022Wradlib/wradlib:V1.15.0}, \textit{xarray} \citep{Hoyer2022Xarray}, \textit{dask} \citep{DaskDevelopmentTeam2016Dask:Scheduling} and \textit{zarr} \citep{Miles2020Zarr-developers/zarr-python:V2.4.0}. The confusion matrix was done with the help of \textit{scikit-learn} \citep{Pedregosa2011Scikit-learn:Python}. The majority of the plots are produced with \textit{matplotlib} \citep{Caswell2019Matplotlib/matplotlib:V3.1.1} unless specified otherwise. The deep learning library used is \textit{Pytorch} \citep{Paszke2019PyTorch:Library} wrapped within \textit{Pytorch-Lightning} \citep{Falcon2020PyTorchLightning/pytorch-lightning:Release}. The computations partly happend on the Vienna Scientific Cluster\footnote{\hyperlink{https://vsc.ac.at//systems/}{https://vsc.ac.at//systems/}} 4 (VSC4) compute nodes, partly on VSC3 using the GPU nodes and partly on Google Compute Cloud.